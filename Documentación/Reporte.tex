\documentclass{article}

\usepackage[spanish]{babel}
\usepackage[letterpaper,top=2cm,bottom=2cm,left=3cm,right=3cm,marginparwidth=1.75cm]{geometry}

% Useful packages
\usepackage{amsmath}
\usepackage{graphicx}
\usepackage{multicol}
\usepackage[colorlinks=true, allcolors=blue]{hyperref}

\title{Reporte Proyecto Final}
\author{José Gerardo González Suárez, Allison Pamela Lagos Tlahuice}

\begin{document}
\maketitle
\begin{abstract}
En las siguientes páginas se documenta el desarrollo del proyecto final de programación, que
consiste en la creación de un programa en C que resuelva un problema presente en algunas de 
las otras materias que cursamos.
\end{abstract}

\section{Índice}
\begin{index}
1. jskjadkjhadkjhadsj
\end{index}

\section{Introducción}
Buscamo un problema que resolver con programación en C, y decidimo hacer un programa que resuelva una ecuació diferencial.
Pero no cualquier ecuación diferencial. En la materia de Ecuaciones Dferenciales tenemos un proyecto experimental en el que
se observa la descomposicion de materia orgánica en un ambiente controlado. El modelo matemático que describe este fenómeno es la
ecuación de decaimiento exponencial, que es una ecuación ordinaria de primer orden.

\section{Objetivo general}
\section{objetivo específicos}

\section{Alcance del proyecto}

\section{Desarrollo}

\section{Fragmentos de código con expliación}

\section{Resultados obtenidos}
\section{conclusiones}
\section{Bibliografía}

\end{document}

